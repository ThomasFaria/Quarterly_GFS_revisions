
% Gemini theme
% https://github.com/anishathalye/gemini

\documentclass[final]{beamer}

% ====================
% Packages
% ====================

\usepackage[T1]{fontenc}
\usepackage{lmodern}
\usepackage[size=custom,width=120,height=72,scale=1.0]{beamerposter}
\usetheme{gemini}
\usecolortheme{mit}
\usepackage{graphicx}
\usepackage{booktabs}
\usepackage{tikz}
\usepackage{pgfplots}
\pgfplotsset{compat=1.14}
\usepackage{anyfontsize}
\usepackage{dsfont}

% ====================
% Lengths
% ====================

% If you have N columns, choose \sepwidth and \colwidth such that
% (N+1)*\sepwidth + N*\colwidth = \paperwidth
\newlength{\sepwidth}
\newlength{\colwidth}
\setlength{\sepwidth}{0.025\paperwidth}
\setlength{\colwidth}{0.3\paperwidth}

\newcommand{\separatorcolumn}{\begin{column}{\sepwidth}\end{column}}

% ====================
% Title
% ====================

\title{How well-behaved are revisions to quarterly fiscal data in the euro area?}

\author{Krzysztof Bankowski \inst{1} \and Thomas Faria \inst{2} \and Robert Schall \inst{3}}

\institute[shortinst]{\inst{1} European Central Bank \samelineand \inst{2} Insee \samelineand \inst{3} European University Institute}

% ====================
% Footer (optional)
% ====================
 
\footercontent{
  \href{https://thomasfaria.github.io/Quarterly_GFS_revisions/}{\url{https://thomasfaria.github.io/Quarterly_GFS_revisions/}} 
  \hfill
  NTTS Conference 2023, Brussels
  }
% (can be left out to remove footer)

% ====================
% Logo (optional)
% ====================

% use this to include logos on the left and/or right side of the header:
\logoright{
  \includegraphics[width=4.5cm, height=4.5cm]{Logo_Insee.png}
  \hspace*{0.75cm}
  \includegraphics[width=4.5cm, height=4.5cm]{Logo_European_Central_Bank.png}}
\logoleft{\includegraphics[width=0.3\linewidth]{QRCODE.pdf}}

% ====================
% Body
% ====================

\begin{document}

\begin{frame}[t]
\begin{columns}[t]
\separatorcolumn

\begin{column}{\colwidth}

  \begin{alertblock}{Summary}
    \textbf{Objective:}
    \begin{itemize}
      \item Determine how well-behaved fiscal revisions are, especially by contrasting them with macro revisions.
    \end{itemize}

    \textbf{Results:}
    \begin{itemize}
      \item Creation a real-time fiscal quarterly dataset for the euro area countries.
      \item We derive a broad set of statistics that allow us to assess the properties of the revisions.
      \item Fiscal revisions, like macro revisions, DO NOT satisfy desirable properties expected from well-behaved revisions.
      \item When contrasted with macro revisions, fiscal revisions are quite comparable (most notably after 2014).
    \end{itemize}

  \end{alertblock}

  \begin{block}{Motivation}

    \begin{itemize}
      \item Most macroeconomic data are subject to \textbf{revisions}.
      \item Researchers and policy-makers have no choice rather than \textbf{recognising and understanding} data revisions.
      \item \textbf{Often heard view} is that fiscal revisions are particularly bad.
      \item Well-behaved revisions ought to carry certain \textbf{characteristics}, according to the relevant literature:
      \begin{itemize}
        \item zero bias
        \item little dispersion 
        \item no predictability in real time
      \end{itemize}
      \item (Our study is a comprehensive analysis of revisions to \textbf{quarterly fiscal data in the euro area}.)
    \end{itemize}
  \end{block}

  \begin{block}{Definitions}
    \begin{itemize}
      \item \textbf{Final revision} for a quarter $t$: $r_t^f =. x_t^f - x_t^1$
      \item Final value $x_t^f$ and we define it as October T+1 release.
      \item \textbf{Intermediate revisions} $r_t^i = x_t^{i+1} - x_t^i$ make up for final revisions.
      \item All revisions are based on \textbf{annual growth differences}.
      \item Our dataset relies on quarterly \textbf{GFS} supplemented with \textbf{MNA}.
  \end{itemize}
  \end{block}

  \includegraphics[width=1\linewidth]{poster-plots_files/figure-pdf/fig-dataoverview1-1.pdf} 

\end{column}

\separatorcolumn

\begin{column}{\colwidth}

  \begin{block}{Real-time quarterly fiscal dataset}
    \begin{itemize}
      \item Our dataset relies on quarterly \textbf{GFS} (Government Finance Statistics) supplemented with \textbf{MNA} (Main National Accounts).
      \item It includes 15 \textit{fiscal} series and 6 \textit{macroeconomic} series. 
      \item Our dataset consists of 65 quarterly vintages taking place since January 2007 to February 2023. We restricted our analysis to pre-covid period (i.e. final values for 2019).
      \item We are committed to open science and have made both the dataset and the replicatory files are \textbf{open source}. Scan the QR code to access them!
    \end{itemize}

    \includegraphics[width=1\linewidth]{poster-plots_files/figure-pdf/fig-revisions_over_time-1.pdf} 


  \end{block}

  \begin{block}{Unconditional properties of revisions}
    \includegraphics[width=1\linewidth]{poster-plots_files/figure-pdf/fig-RevGItems_post2014-1.pdf} 

    \begin{itemize}
      \item We look at a set of summary statistics including the mean revision (\textbf{MR}), the maximum and the minimum revision (\textbf{MAX} and \textbf{MIN}), the mean absolute revision (\textbf{MAR}), the root-mean-square revision (\textbf{RMSR}) and, finally, the noise-to-signal ratio (\textbf{N2S}).
      \item Almost all variables we consider in the analysis are associated with a \textbf{positive bias}, as judged by the MR statistic. \item Other measures, namely MIN-MAX range, MAR and RMSR, indicate that the revisions tend to have \textbf{large dispersion}.
      \item The degree of \textit{misbehaviour} of fiscal revisions \textit{after 2014} is comparable to \textit{macroeconomic} series revisions.
    \end{itemize}

  \end{block}

\end{column}
\separatorcolumn


\begin{column}{\colwidth}



  \begin{block}{Forecastability of revisions}

    \textbf{Key question}: is the complete model better than the naive model? In other words, does any information available at the time of the initial release have predictive power?

    \textbf{Naive model}: 
    \[
    r_{t,m}^{f} = \epsilon_{t,m}
    \]
    %\bigskip{}
    \textbf{Complete model}: 
    \[
      r_{t,m}^{f} = \sum_{m=1}^{9}\beta_{m}C_{m} 
    + \sum_{j=1}^{4} \gamma_{j}Q_{t}^{j} 
    + \omega\mathds{1}_{\left[t\geq2014Q2\right]}
    + \delta x_{t,m}^{1} 
    + \sum_{i=1}^{S} \rho_{i}\left(x_{t-i,m}^{i+1}-x_{t-i,m}^{1}\right)
    + \epsilon_{t,m}
    \]

    \begin{tabular}{lll}
         $C_{m}$ & $\rightarrow$ & country dummies \\
         $Q_{t}^{j}$ & $\rightarrow$ & quarter dummies \\
         $\mathds{1}_{\left[t\geq2014Q2\right]}$ & $\rightarrow$ & ESA 2010 introduction dummy \\
         $x_{t,m}^{1}$ & $\rightarrow$ & initial release \\
         $\left(x_{t-i,m}^{i+1}-x_{t-i,m}^{1}\right)$ & $\rightarrow$ & past revisions \\
      \end{tabular}

    \vspace*{0.5cm}

    There are two equally effective methods for assessing predictive power: examining the \textbf{RMSE reduction} or conducting the \textbf{joint F-test}.

    \vspace*{0.5cm}
    \textbf{Result}:
    \begin{itemize}
      \item Irrespective of fiscal and macro variables, we found a model that outperforms the naive model, indicating the existence of \textbf{predictability} in real-time revisions. Specifically, we identified that the initial release holds significant predictive power for subsequent revisions.
    \end{itemize} 

  \end{block}

  \begin{block}{Intermediate revisions}
    \includegraphics[width=1\linewidth]{poster-plots_files/figure-pdf/fig-Single_Rev_Path-1.pdf} 

  \end{block}

  \begin{block}{Selected literature}
    \nocite{*}
    \footnotesize{\bibliographystyle{plain}\bibliography{poster}}

  \end{block}

\end{column}

\separatorcolumn
\end{columns}
\end{frame}

\end{document}

